\documentclass[nociteref]{newsiambook}

\usepackage{hyperref}
\usepackage{import}
\usepackage{amsmath, amsfonts, amscd, amssymb}
\usepackage{mathtools}
\usepackage{epsfig}
\usepackage{graphicx}
\usepackage{url}
\usepackage{mathrsfs}
\usepackage{makeidx}
\usepackage{multicol}
\usepackage{color}
\usepackage{verbatim}
\usepackage{listings}
\usepackage{pseudocode}
\usepackage{framed}
\usepackage{float}
\usepackage{paralist}
\usepackage{caption, subcaption}

%------------------------------------------------------------------------------%
% command.tex                                                                  %
% This file contains the various environments and other misc. commands         %
%------------------------------------------------------------------------------%

%counter for problems. reset each chapter
\newcounter{problemnum}[chapter]


\newcommand{\objective}[1]{\vspace{5mm}{\bf Lesson Objective: } \emph{#1} \vspace{5mm}}
\renewcommand{\chaptername}{Lab}
\renewcommand{\bibname}{References}

\newcommand{\lab}[3]{\chapter[#3]{#1: #2}}

% Various commands that make life easier
\newcommand{\argmax}{\mbox{argmax}}
\newcommand{\indicator}[1]{\mathbbm{1}_{\left[#1\right] }}
\providecommand{\abs}[1]{\left\lvert#1\right\rvert}
\providecommand{\norm}[1]{\left\lVert#1\right\rVert}
\providecommand{\set}[1]{\lbrace#1\rbrace}
\providecommand{\setconstruct}[2]{\lbrace#1:#2\rbrace}
\DeclareMathOperator{\res}{res}           % Residue
\DeclareMathOperator{\Res}{Res}           % Residue

\newenvironment{amatrix}[1]{%
\left(\begin{array}{@{}*{#1}{c}|c@{}}
}{%
\end{array}\right)
}

\newenvironment{dmatrix}[2]{%
\left(\begin{array}{@{}*{#1}{c}|*{#2}{c}@{}}
}{%
\end{array}\right)
}

\newenvironment{pseudo}[2]
    {\begin{pseudocode}[shadowbox]{#1}{#2}}
    {\end{pseudocode}}

\newenvironment{problem}{\begin{shaded}\begin{problemnum}}{\end{problemnum}\end{shaded}}

\newtheoremup{problemnum}{Problem}
\definecolor{shadecolor}{gray}{0.90}

\newcommand{\li}[1]{\lstinline[style=python]!#1!}

\newcommand{\ipt}[2]{\langle #1,#2 \rangle}
\newcommand{\ip}{\int_{-\infty}^{+\infty}}

\renewcommand{\ker}[1]{\mathcal{N}(#1)}
\newcommand{\ran}[1]{\mathcal{R}(#1)}

\include{include}

\makeindex

\begin{document}

%-------------------------------------------------------------



%----------------------------------------------------------------
%Book cover and Front matter
\thispagestyle{empty}
\begin{center}
{\huge \bf Applied Mathematics} \\ and \\ {\huge \bf Computing} \\
\vspace{5mm}
{\Large Volume I}
\vspace{20mm}

\includegraphics[scale = .25]{Cover}
\end{center}
\frontmatter

\subimport{./}{contributors.tex}

%------------------------------------------------------------------
%The preface, which will presumably be longer in the future

\begin{thepreface}
This lab manual is designed to accompany the textbook \emph{Foundations of Applied Mathematics} by Humpherys and Jarvis.

\vfill
\copyright{This work is licensed under the Creative Commons Attribution 3.0 United States 
License.  You may copy, distribute, and display this copyrighted work only if you give 
credit to Dr.~J.~Humpherys. All derivative works must include an attribution to Dr.~J.~Humpherys as the owner of this work as well as the web address to 
\\\centerline{\url{https://github.com/byuimpact/numerical_computing}}\\ as the original source of 
this 
work.\\To view a copy of the Creative Commons Attribution 3.0 License, 
visit\\\centerline{\url{http://creativecommons.org/licenses/by/3.0/us/}} or send a letter to 
Creative Commons, 171 Second Street, Suite 300, San Francisco, California, 94105, USA.}

\vfill
\centering\includegraphics[height=1.2cm]{by}
\vfill
\end{thepreface}
%-----------------------------------------------------------------

\setcounter{tocdepth}{1}
\tableofcontents

\mainmatter

\part{Python Essentials}
\subimport{./Python/GettingStarted/}{GettingStarted}
\subimport{./Python/Arrays/}{Arrays}
\subimport{./Algorithms/Matrices_Complexity/}{Matrices_Complexity}
\subimport{./Python/Vectorization/}{Vectorization}

\part{Linear Transforms}
\subimport{./Algorithms/ChangeBasis/}{ChangeBasis}
\subimport{./Applications/MarkovGraph/}{MarkovGraph_C}
\subimport{./Algorithms/LUdecomposition/}{LUdecomposition}
\subimport{./Applications/Leontief/}{Leontief_C}

\part{Inner Product Spaces}
\subimport{./Algorithms/QR/}{QR_C}
\subimport{./Applications/CorrCovariance/}{CorrCovariance}
\subimport{./Algorithms/CanonTransform/}{CanonTransform} 
\subimport{./Applications/LeastSquares/}{LeastSquares}

\part{Spectral Theory}
\subimport{./Algorithms/Givens/}{Givens}
\subimport{./Applications/ImageSegment/}{ImageSegment}
\subimport{./Algorithms/EigSolver/}{Eig_C}
\subimport{./Applications/SVD/}{SVD}

\part{Metric Spaces}
\subimport{./Python/Profiling/}{Profiling}
\subimport{./Python/cython/}{cythonlab}
\subimport{./Python/matplotlib/}{matplotlib}
\subimport{./Python/sympy/}{sympylab}

\part{Differentiation}
\subimport{./Algorithms/NumDeriv/}{FiniteDiff_C}
\subimport{./Applications/BeamBuckle/}{Beams_C}
\subimport{./Algorithms/MultiDeriv/}{FiniteDiff2_C}
\subimport{./Applications/ImgFilters/}{ImgFilters}

\part{Contractions}
\subimport{./Applications/NewtonsMethod/}{Newton_C}
\subimport{./Applications/Julia/}{Julia}

\part{Convex Analysis}
\subimport{./Algorithms/linesweep/}{linesweep}
\subimport{./Applications/voronoi/}{voronoi}

\part{Riemann-Darboux Integration}
\subimport{./Algorithms/GaussQuad/}{GaussQuad}
\subimport{./Applications/MC/}{MC}

\part{Curves and Surfaces}
\subimport{./Algorithms/Splines/}{bezier}
\subimport{./Algorithms/Splines/}{bsplines}
\subimport{./Applications/NURBS/}{NURBS_C}

\part{Differential Forms}

\part{Complex Integration}
\subimport{./Applications/ConfMaps/}{ConfMaps}
\subimport{./Applications/RiemannSphere/}{RiemannSphere_C}
\subimport{./Applications/ComplexIntegration/}{ComplexIntegration1}
\subimport{./Applications/ComplexIntegration/}{ComplexIntegration2}

\part{Spectral Calculus}
\subimport{./Algorithms/Krylov/}{Krylov}
\subimport{./Applications/MarkovPerron/}{MarkovPerron}
\subimport{./Algorithms/Drazin/}{Drazin}

\part{Perturbation Theory}

\part{Transcient Behavior}

\part{Algebraic Rings}

\part{Appendices}
\subimport{./Python/installguide/}{Win64}

\end{document}
